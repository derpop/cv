

\documentclass[]{deedy-resume-openfont}
\usepackage{fancyhdr}
\usepackage{comment}
 
\pagestyle{fancy}
\fancyhf{}
 
\begin{document}

%%%%%%%%%%%%%%%%%%%%%%%%%%%%%%%%%%%%%%
%
%     LAST UPDATED DATE
%
%%%%%%%%%%%%%%%%%%%%%%%%%%%%%%%%%%%%%%
\lastupdated

%%%%%%%%%%%%%%%%%%%%%%%%%%%%%%%%%%%%%%
%
%     TITLE NAME
%
%%%%%%%%%%%%%%%%%%%%%%%%%%%%%%%%%%%%%%
\namesection{Devin}{Seth}{
  \urlstyle{same}
  \href{mailto:sethd@rpi.edu}{sethd@rpi.edu} \;|\;
  609-212-9150 \;|\;
  \href{https://github.com/derpop}{github.com/derpop} \;|\;
  \href{https://https://www.linkedin.com/in/devinseth/}{https://www.linkedin.com/in/devinseth}
}

%%%%%%%%%%%%%%%%%%%%%%%%%%%%%%%%%%%%%%
%
%     COLUMN ONE
%
%%%%%%%%%%%%%%%%%%%%%%%%%%%%%%%%%%%%%%

\begin{minipage}[t]{0.33\textwidth}

\section{Education}
\subsection{Rensselear Polytechnic Institute}
\descript{B.S. Physics \& Computer Science (Dual)}
\location{Expected May 2028 | Troy, NY}
\sectionsep

\subsection{The Hun School of Princeton}
\descript{High School Diploma}
\location{June 2024 | Princeton, NJ}

\begin{comment}
\section{Technical Summary}
\begin{itemize}\itemsep 0.2em
  \item Computational astrophysics: N-body simulation (EXP), stellar evolution (MESA), data analysis.
  \item Systems/automation: OCR-driven inventory workflows, data analysis.
  \item HPC practices: multi threading, MPI, Linux tooling, Git/GitHub.
\end{itemize}
\sectionsep
\end{comment}


\section{Skills}
\textbf{Languages \& Tools:} Python, C++, Java, C\#, MATLAB, \LaTeX, Git/GitHub \\
\textbf{Scientific Computing:} NumPy, SciPy, Matplotlib, Astropy, Mathematica \\
\textbf{Astrophysics/HPC:} EXP ($N$-body), MESA, MPI, Linux, Multithreading
\sectionsep

\section{Coursework}
Data Structures \textbullet{} Linear Algebra \textbullet{} Differential Equations \\
Electromagnetic Theory \textbullet{} Introduction to Quantum Physics
\sectionsep

\section{Awards}
\begin{tabular}{r@{\hspace{0.6em}}l}
2023 & INSPIRE Award (FTC) \\
2024 & Dean’s List — {RPI} \\
\end{tabular}
\sectionsep


\section{Projects}
\runsubsection{Efficient Octree-Based N-Body Simulation}
\location{Nov 2024}
\begin{itemize}\itemsep -0.3em
  \item Built a Barnes–Hut (octree) $N$-body simulator to model galaxy interactions, reducing complexity from $O(N^2)$ to $O(N\log N)$.  
  \item Implemented in C\# with multithreading; supports real-time 3D visualization
  \item Designed for scalability (2,000–10,000+ particles), with planned GPU acceleration
  \item Code: \href{https://github.com/derpop/GravitySimulation}{github.com/derpop/GravitySimulation}  
\end{itemize}


\end{minipage} 
\hfill
\begin{minipage}[t]{0.66\textwidth} 



%%%%%%%%%%%%%%%%%%%%%%%%%%%%%%%%%%%%%%
%     RESEARCH
%%%%%%%%%%%%%%%%%%%%%%%%%%%%%%%%%%%%%%


\section{Work Experience}
\runsubsection{Evolution of Smooth (EOS)}
\descript{| R\&D Engineer}
\location{Jun 2025 – Aug 2025}
\begin{itemize}\itemsep -0.4em
  \item Designed and deployed an automated, inventory tracking system.
  \item Built an OCR pipeline to parse shipping tags and populate order records, reducing manual errors.
  \item Supported \textasciitilde{}8{,}000 SKUs; processed $>$100{,}000 items for e-commerce channels.
  \item Increased order accuracy rate to $\sim$99\% by eliminating mis-entered records.

\end{itemize}
\section{Clubs}
\runsubsection{RXPI (Rensselaer Experimental Propulsion)}
\descript{| Systems Programmer}
\location{Nov 2024 – Present}
\begin{itemize}\itemsep -0.3em
  \item Contributed to the design and operation of a liquid-fueled rocket engine (kerosene / nitrous oxide).  
  \item Programmed microcontroller systems to manage engine startup, shutdown, and communication protocols.  
  \item Developed and maintained control software; open-source code available at: \href{https://github.com/Rensselaer-Spaceflight-Society/Rocket-Engine}{github.com/Rensselaer-Spaceflight-Society/Rocket-Engine}.  
\end{itemize}


\section{Research}
\runsubsection{Stellar Evolution of Main Sequence Stars Powered Only by the PP Chain}
\descript{| Researcher \& Author}
\location{May 2024 – Oct 2024 \; | \; East Windsor, NJ}
\begin{itemize}\itemsep -0.3em
  \item Independently designed and executed a stellar evolution study using MESA, restricting nuclear reactions to the proton–proton (PP) chain. 
  \item Modeled stars from $1$–$2\,M_\odot$ to compare lifetimes, luminosities, and structural properties with standard (PP + CNO + triple-alpha) networks. 
  \item Found counterintuitive results: PP-only stars were \textit{longer lived and more luminous}, with hotter, denser cores and altered shell-burning behavior. 
  \item Proposed convective overshoot as a mechanism extending PP-chain main sequence lifetimes, connecting results to models of Population III stars. 
  \item Published results in the \textit{Journal of Student Research} \href{https://doi.org/10.47611/jsrhs.v13i1.6021}{DOI}, 300+ downloads.
\end{itemize}
\sectionsep

\runsubsection{Heidi Newberg Research Group}
\descript{| Undergraduate Researcher}
\location{Jan 2025 – Present \; | \; Troy, NY}
\begin{itemize}\itemsep -0.3em
  \item Develop high-resolution $N$-body models of the Milky Way and Large Magellanic Cloud (LMC) to investigate tidal interactions and stellar stream formation. 
  \item Simulate galactic collisions with $>$5{,}000{,}000 particles using the EXP code on HPC resources (dual-socket Xeon, 48 threads, 187\,GiB RAM). 
  \item Analyze resulting tidal streams and disk substructure to test competing dark-matter halo models of the Milky Way. 
  \item Generate and validate simulation outputs to serve as input datasets for a new developmental feature on \textit{Milkyway@home}. 
  \item Serve as the most senior undergraduate in the project, maintaining continuity of methods and mentoring new team members. 
\end{itemize}


%%%%%%%%%%%%%%%%%%%%%%%%%%%%%%%%%%%%%%
%     AWARDS
%%%%%%%%%%%%%%%%%%%%%%%%%%%%%%%%%%%%%%

\end{minipage} 
\end{document}  \documentclass[]{article}
